\section{Introduction}\label{Sec: introducao}

Irregular satellites revolve around giant planets at large distances in eccentric, highly inclined and frequently retrograde orbits. Because of these peculiar orbits, it is largely accepted that these objects did not form by accretion around their planet, but were captured in the early solar system \citep{Sheppard2005}.

There is no consensus for a single model explaining where the irregular satellites were formed. \cite{Cuk2004} showed that the progenitor of the Himalia group may have originated in heliocentric orbits similar to the Hilda asteroid group. \cite{Sheppard2005} stated that the irregular satellites may be some of the objects that were formed within the giant planets region.

\cite{Grav2003} and \cite{Grav2007} showed that the irregular satellites from the giant planets have their colors and spectral slopes similar to C-, D- and P-type asteroids, Centaurs and trans-neptunian objects (TNOs). This suggests that they may have come from different locations in the early solar system.

\cite{Sheppard2005} and \cite{Jewitt2007} also explored the possibility that the irregular satellites originated as comets or TNOs. TNOs are highly interesting objects that, due to their large heliocentric distances, may be highly preserved with physical properties similar to those they had when they were formed \citep{Barucci2008}. This is even more true for the smaller objects, since in principle larger sizes favour physical differentiation processes in the body and vice-versa. However, due to the distance, the smaller TNOs from this region are more difficult to observe. Thus, if irregular satellites - or at least a few of them - do share a common origin with small TNOs, and since these objects are situated at much closer heliocentric distances now, this gives a unique chance of observing and studying representatives of this specific TNO population in much greater detail than could ever be possible by direct observation of this population in the Kuiper Belt. 

Phoebe is the most studied irregular satellite. \cite{Clark2005} suggest that its surface is probably covered by material of cometary origin. It was also stated by \cite{Johnson2005} that if the porosity of Phoebe is 15\%, Phoebe would have an uncompressed density similar to those of Pluto and Triton.

In order to obtain precise fundamental physical parameters like size and shape, thus constraining the albedo and in a broader sense also the composition for the irregular satellites and therefore to contribute to the study of their origin, we aim at observing stellar occultations, which provide  more accurate results than other ground-based techniques \citep{Sicardy2011, Ortiz2012, Braga-Ribas2014}. For that, reliable predictions of stellar occultations by these satellites are most needed.

We present in this paper stellar occultation predictions for the 8 major irregular satellites of Jupiter (Himalia, Elara, Pasiphae, Lysithea, Carme, Ananke, Sinope and Leda), Phoebe of Saturn and, Nereid and Triton of Neptune. Phoebe, being the most studied object with a good measured size, can be used to calibrate and evaluate the technique for similar objects.

Triton is an uncommon satellite. Its orbit is retrograde and inclined, but quasi-circular and very close to the planet compared to the irregular ones. Because Triton's orbit size is very small and its precession is not dominated by Solar perturbations, Triton is frequently excluded from the irregular satellites' class, but still studied together by many authors \citep{Sheppard2005, Jewitt2007}. Similarly to the irregular satellites, Triton was probably captured in the early solar system and may have the same origin as the TNOs \citep{Agnor2006}. However, Triton is bigger than the irregular satellites by an order of magnitude and has an atmosphere. The main motivation to study Triton by stellar occultations is to understand the evolution of its atmosphere due to Triton's complicated and extreme seasonal cycle \citep{McKinnon2007, Elliot_2000}. For all these reasons, Triton was also included as a target in this work.

Excluding Triton \citep{Olkin1997, Elliot_2000}, no observation of a stellar occultation by an irregular satellite was published up to date. Since their estimated sizes are very small (see Table \ref{Tab: satellite-diameter}), this may have discouraged earlier tries. But, in fact, given their relatively closer distances as compared to TNOs and Centaurs, and considering the current precision of their ephemeris and of star positions, we can now reliably predict the exact location and instant where the shadow of the occultation will cross the Earth. For instance, Himalia, supposedly the largest irregular satellite of Jupiter has an estimated size of 150 km \citep{Porco2003}, which is equivalent to an apparent size of about 40 mas (milliarcseconds). Thus, in this case, if the accumulated error (2 sigma or 95\% confidence level) of ephemeris and star position be around 70 mas, we have a probability of about 30\% of observing a stellar occultation, which is quite satisfactory today (see discussion in section \ref{Sec: discussion}).

%\begin{table}
%\caption{\label{Tab: satellite-diameter} Estimated diameter of the satellites and correspondent apparent diameter}
%\begin{centering}
%\begin{tabular}{lccc}
%\hline  \hline
%\multicolumn{4}{c}{Diameter of the satellites} \tabularnewline
%Satellite  & mas\tablefootmark{a}  & km & Ref. \tabularnewline
%\hline
%Ananke & 8 & 29 & 1 \tabularnewline
%Carme & 13 & 46 & 1 \tabularnewline
%Elara & 24 & 86 & 1 \tabularnewline
%Himalia & 41 & $(150\times120) \pm 20$\tablefootmark{b} & 2 \tabularnewline
%Leda & 5 & 20 & 1 \tabularnewline
%Lysithea & 10 & 36 & 1 \tabularnewline
%Pasiphae & 17 & 62 & 1 \tabularnewline
%Sinope & 10 & 37 & 1 \tabularnewline
%\hdashline
%Phoebe & 32 & $212 \pm 1.4$\tablefootmark{b} & 3 \tabularnewline
%\hdashline
%Nereid & 15 & $340 \pm 50$\tablefootmark{c} & 4 \tabularnewline
%Triton & 124 & $2707 \pm 2.0$\tablefootmark{c} & 5 \tabularnewline
%\hline
%\end{tabular}
%\tablebib{
%(1) \cite{Rettig2001}; (2) \cite{Porco2003}; (3) \cite{Thomas2010}; (4) \cite{Thomas1991}; (5) \cite{Thomas2000}.}
%\end{centering}
%\tablefoottext{a}{Using a mean distance from Jupiter of 5 AU, from Saturn of 9 AU and from Neptune of 30 AU.}
%\tablefoottext{b}{From Cassini observations.}
%\tablefoottext{c}{From Voyager-2 observations.}
%\par
%\end{table}

\cite{GomesJunior2015} obtained 6523 suitable positions for 18 irregular satellites between 1992 and 2014 with an estimated error in the positions of about 60 to 80 mas. For some satellites the number of positions obtained is comparable to the number used in the numerical integration of orbits by the JPL \citep{Jacobson2012}. They pointed out that the ephemeris of the irregular satellites have systematic errors that may reach 200 mas for some satellites. For an object at the distance of Jupiter, this represents an error larger than 700 km in the shadow path. Using the positions obtained by \cite{GomesJunior2015} we produced a specific short-time ephemeris for the satellites of Jupiter and Phoebe to better predict stellar occultations by these objects.

Since 2009 many successful observations of stellar occultations by TNOs have been reported in the literature \citep{Elliot2010, Sicardy2011, Ortiz2012, Braga-Ribas2013}, the main disadvantages in their prediction being large heliocentric distances and ephemeris error, facts somewhat compensated for the larger diameters involved. In contrast to TNOs, the irregular satellites have much better ephemeris because the orbits of their host planets are better  known, their observational time span is much wider and covers many orbital periods. Moreover, the irregular satellites are much closer to Earth which implies in a much smaller shadow path error in kilometers. These advantages may be somewhat balanced by the smaller sizes estimated for the irregular satellites.Thus, in comparison, the chances for a successful observation of an stellar occultation by an irregular satellite should be considered at least also as good as those by TNOs.

%From what we have just presented, we can think that the observation of an stellar occultation by an irregular satellite should be likelier than occultations by TNOs. The orbits of their host planets are well known and their observation time-span covers many orbital periods, contrary to TNOs.
%Unlike stellar occultations by TNOs, which nevertheless have been proved effective, the observation of an occultation by a irregular satellite is, in principle, more favorable. The orbits of their host planets are well known and these satellites have been observed completing already many orbital periods around them, thus presenting better ephemeris than TNOs. 
%Moreover, the irregular satellites are closer to Earth which means a minor localization error in kilometer.

In section \ref{Sec: integration} we show the building of the production of the new ephemeris. In section \ref{Sec: predictions}, we present the predictions of the stellar occultations by irregular satellites and how they were made. Some tests made to check the accuracy of the predictions are presented in section \ref{Sec: testes} and the final discussion is presented in section \ref{Sec: discussion}.