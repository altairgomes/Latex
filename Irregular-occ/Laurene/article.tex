\documentclass[traditabstract]{aa}
\usepackage{txfonts}
\usepackage{graphicx}
\usepackage{longtable}
\usepackage{natbib}
\usepackage{color}
\bibpunct{(}{)}{;}{a}{}{,} % to follow the A&A style
\begin{document}


\title{Dynamical parameter determinations in Pluto's system}
\subtitle{Expected constraints from the New Horizons mission to Pluto}

\author{L. Beauvalet\inst{1}
\and V. Lainey \inst{1}
\and J.-E. Arlot\inst{1}
\and R. P. Binzel\inst{2}\fnmsep\inst{1}
}

\institute{Institut de M\'ecanique C\'eleste et de Calcul des \'Eph\'em\'erides-Observatoire de Paris, UMR 8028 du CNRS, Universit\'e Pierre et Marie Curie, Universit\'e Lille 1, 77 avenue Denfert Rochereau, 75014 PARIS, France \\
\and
Department of Earth, Atmospheric, and Planetary Sciences, Massachusetts Institute of Technology, 77 Massachusetts Avenue, Cambridge, MA 02139 USA
}
%\date{Received ; accepted}

\abstract
{
Pluto is the multiple system that has been observed the longest. Yet, the masses of its smallest satellites, Nix and Hydra, which were discovered in 2005, are still imprecisely known, because of the short time span and number of available observations. We present a numerical model that takes into account the second order gravity fields and Pluto's orbital motion in the solar system. We investigated the dynamical parameters that may be reliably determined today. We also assessed the possible improvements on the parameter uncertainties with the future increase of observations, including the New Horizons mission. Fitting our model to simulated data, we show that the precision of observations prevents the quantification of the polar oblateness $J_2$ and equatorial bulge $c_{22}$ of Pluto and Charon. Similarly, we show that the masses are on the detection limit.  In particular, unless 25 observations are made every year, the mass of Nix may be constrained with confidence only with New Horizons data. Hydra's mass will only be constrained by the probe. The recent discovery of P4 might change this situation, but our knowledge of this object is still too vague to draw any conclusion.
}

   \keywords{Celestial mechanics -- Planets and satellites: fundamental parameters -- planets and satellites: individual: Pluto -- Methods: numerical -- Kuiper belt: general
               }

   \maketitle





\section{Special tailored Ephemerides (STE) for Jupiter irregular satellites}
\newpage
\subsection{Observations used}

 
The last observations used to develop JPL current ephemeris of the irregular satellites of Jupiter were obtained in 2012 \citep{Jacobson2012}. As a result, the errors on the residuals of recent observations are too important for the ephemeris to accurately predict stellar occultations without anny corrections. In order to have efficient ephemerides for predicting the stellar occultations by irregular satellites, we need predictions based on recent observations. This is why we decided to develop our own ephemeris based on the observations published in \cite{GomesJunior2015}.

{\color{red}Here comes your paragraphs from \emph{Correction of the ephemeris} about your observations from Gomes-Junior 2015 and how they present systematic errors. Also, the little blurb I write just before will need to be reread and edited (because editing others is sooooo easier and more efficient than editing oneself)}

\newpage
\subsection{Description of the model}


Our numerical model describes the dynamical evolution of the irregular satellites of Jupiter in a jovicentric reference frame. The satellites are submitted to the influence of the Sun and the rest of the solar system, as well as those of the Galilean satellites and the first harmonics of Jupiter's gravity field. The axis of the reference frame are those of the equatorial reference frame J2000. 

We use the following notations: \begin{itemize}
\item $i$ and $l$ one of the irregular satellites of Jupiter
\item $J$ Jupiter 
\item $j$ another body of the Solar System
\item $M_j$ the mass of the $j$ body, not an irregular satellite
\item $m_i$ the mass of the irregular satellite $i$
\item $\vec{r_i}$ the position of the $i$ body with respect to the barycentre of Jupiter System
\item $r_{ij}$ the distance between bodies $i$ and $j$  
\item $R_J$ the radius of Jupiter
\item $J_n$ the dynamic polar oblateness of the nth order for Jupiter's gravity field
\item $U_{\bar{l}\hat{J}}$ potential generated by the oblateness of Jupiter on the satellite $l$
\item $\Phi_i$ is the inclination of the $i$ satellite with respect to Jupiter's equator.
\end{itemize}

For an irregular satellite $i$, under the gravitational influence of Jupiter, the $\mathcal{N'}-1$ other irregular satellites, the regular Jovian satellites and the rest of the Solar System ($\mathcal{N}$ bodies), the equation of motion is:
\begin{equation}\begin{array}{ll}

\ddot{\vec{r_i}}= & \displaystyle -GM_J\frac{\vec{r_J}-\vec{r_i}}{r_{iJ}^3}-\sum_{l=1,l\neq i}^\mathcal{N'}Gm_l\frac{\vec{r_l}-\vec{r_i}}{r_{il}^3}\\
&\displaystyle -\sum_{j=1}^\mathcal{N}GM_j \left(\frac{\vec{r_j}-\vec{r_i}}{r_{ij}^3} - \underbrace{\frac{\vec{r_j}-\vec{r_J}}{r_{Jj}^3}}_\mathtt{undirect perturbations} \right)\\
 & \displaystyle +GM_J \nabla U_{\bar{l}\hat{J}} -\underbrace{\sum_{l=1}^\mathcal{N} Gm_l\nabla U_{\bar{l}\hat{J}}}_\mathtt{undirect perturbations}
\end{array}
\end{equation}

where the oblateness potential seen by the body $i$ because of Jupiter is:
\begin{equation}\begin{array}{ll}

U_{\bar{l}\hat{J}}=&\displaystyle -\frac{R_J^2 J_2}{r_{iJ}^3}\left(\frac{3}{2}\sin^2 \Phi_i-\frac{1}{2}\right)\\ &\\ & 
\displaystyle-\frac{R_J^4 J_4}{r_{iJ}^5}\left(\frac{35}{8}\sin^4 \Phi_i-\frac{15}{4}\sin^2 \Phi_i+\frac{3}{8}\right)\\
& \\
&\displaystyle-\frac{R_J^6 J_6}{r_{iJ}^7}\left(\frac{231}{16}\sin^6 \Phi_i-\frac{315}{16}\sin^4 \Phi_i+\frac{105}{16}\sin^2 \Phi_i-\frac{5}{16}\right)

\end{array}
\end{equation}

The expressions of $\nabla_lU_{\bar{l}\, \hat{i}}$ and $\nabla_i U_{\bar{i}\, \hat{l}}.$ have been developed in \citet{Lainey2004}.The equations of motion are integrated with the numerical integrator RADAU \citep{Everhart1985}. 
Our model was fitted to the observations through a least-squares procedure. The satellites were integrated one dynamical family at a time, to gain computing time, while losing minimum precision. Indeed, the interactions between satellites not belonging to the same dynamical family are negligible considering the short timespan of our integration. 

The initial osculating elements at the origin of integration are presented in Table \ref{sat_ell}, while the residuals are presented in Fig \ref{omcHim} to \ref{}. %don't remember wich one

All the orbits determined for the satellites show satisfying residuals. Yet, one of the satellites has a peculiar situation. Its small number of observations in our set means that its orbit is definitely loosely constrained. 
The residuals are lower than those obtained with JPL ephemeris, but the accuracy of an ephemeris decreases when we get further from the time of observations. The main risk of divergence over time comes from the possible absence of long-term effects when fitting to a short timespan of observations. If that were the case, our ephemeris would diverge too quickly to be of any use. JPL ephemerides are fitted over all the available observations. As a result, they will diverge less quickly than our own. Though they are no longer precise enough for our use, they remain a recious reference to identify whether our own model presents a quick divergence. We compared our ephemeris to the JPL for all the Jupiter satellites we fitted, until 2017. The divergence between 2015 and 2018 is at most \textit{\color{red}insert value here} for \textit{\color{red}insert right satellite name here}. A example of this difference for satellites Himalia and \textit{\color{red}whichever satellite you want, if you want)} is presented in Figure \ref{JPL-STE}.

The obtained ephemeris is hereafter refered as STE, for special-tailored ephemeris. 

\begin{table}
\caption{Initial osculating elements for Jupiter irregular satellites at JD 2451545.0. For Themisto's case, the small number of observations meant that the statistical uncertainty was even greater than the obtained elements, though the fitting proved satisfying. We decided to provide its elements with significant number comparable to the other satellite for information, but left it without uncertainty.  }\label{sat_ell}
\begin{center}
\begin{tabular}{ccccccc}
\hline\hline
Satellite & a (km) & e & incl. (EQJ2000) & $\Omega$ & $\omega$ & $v$ \\ 
\hline
Himalia &   11372100 $\pm$ 500    &    0.166 $\pm$ 0.002      &   45.14 $\pm$ 0.15      &   39.77 $\pm$ 0.19      &   351.48 $\pm$ 0.46      &   97.35 $\pm$ 0.48    \\
Elara &   11741170 $\pm$ 690  &      0.222 $\pm$ 0.002      &   28.64 $\pm$ 0.18      &   68.42 $\pm$ 0.43      &   179.82 $\pm$ 0.56      &   339.08 $\pm$ 0.82  \\
Pasiphae &  23425000  $\pm$ 5000    &     0.379  $\pm$ 0.001       &   152.44 $\pm$ 0.10      &   284.59 $\pm$ 0.21      &   135.96 $\pm$ 0.19      &   236.97 $\pm$ 0.16 \\
Sinope &   22968800 $\pm$ 5200   &     0.316 $\pm$ 0.002      &   157.76 $\pm$ 0.12      &   256.62 $\pm$ 0.55      &   298.38 $\pm$ 0.55      &   167.57 $\pm$ 0.19    \\
Lysithea &   11739900 $\pm$ 1300  &      0.136 $\pm$ 0.004      &    51.12 $\pm$ 0.27     &   5.53 $\pm$ 0.52      &   53.0 $\pm$ 1.5      &   318.9 $\pm$ 2.0   \\
Carme &   24202924 $\pm$ 4800      &  0.242 $\pm$ 0.001      &   147.13 $\pm$ 0.10      &   154.01 $\pm$ 0.25      &   47.90 $\pm$ 0.29      &   234.41 $\pm$ 0.19  \\
Ananke &  21683800  $\pm$ 7200  &     0.380 $\pm$ 0.002      &   172.29 $\pm$ 0.20      &   56.9 $\pm$ 1.2      &   123.3 $\pm$ 1.2      &   231.24 $\pm$ 0.21  \\
Leda &   11140300  $\pm$ 4300  &     0.173  $\pm$ 0.007     &   16.15  $\pm$ 0.75    &   272.6  $\pm$ 1.7    &   212.2  $\pm$ 3.6          &   218.8  $\pm$ 3.2  \\
Themisto  & 7393800    &     0.198       &   25.77  &   220.0     &   216.3    &   262.1     \\ \hline
\end{tabular} 

\end{center}
\end{table}
\bibliographystyle{plainnat} 
\bibliography{biblio} % your references Yourfile.bib

\end{document}


















