\section{Discussion} \label{Sec: discussion}

We predict stellar occultations for the period of 2016-2017 for eight irregular satellites of Jupiter: Ananke, Carme, Elara, Himalia, Leda, Lysithea, Pasiphae, and Sinope; one satellite of Saturn: Phoebe; and two satellites of Neptune: Triton and Nereid. The procedure used was the same as that for the prediction of stellar occultations by Pluto and its satellites in \cite{Assafin2010} and by Centaurs and TNOs in \cite{Assafin2012} and \cite{Camargo2014}.

The candidate stars were searched in the UCAC4 catalogue, except for the candidates in 2016 for Triton and Nereid. In this case, we used the WFI catalogue that was generated from observations made with ESO2p2/WFI CCD mosaic that covered the path of Neptune in the sky-plane up to 2016 (see Sec. \ref{Sec: predictions}). From this, a total of 396 events are foreseen. 

In a broader, general sense, the probability of successfully observing an occultation is roughly the ratio of the satellite's radius by the budget error (2 sigma for a 95\% confidence level) of ephemeris and star position. Thus, UCAC4 errors ranging between 20 mas - 50 mas (1 sigma) combined with a mean error (1 sigma) in the JPL ephemeris of 30 mas for Himalia and 150 mas for Leda published in Table 2 of \cite{Jacobson2012} would give 28\%-17\% probability of observing such an event by Himalia and $\approx2$\% for Leda, the smallest irregular satellite in the sample. Observations a few days before the date of occultation predicted may improve the combined errors to 40-80 mas, depending on the magnitude of the objects.

The test made with an occultation expected to happen in March 03, 2015 for Himalia showed that this event would probably have been observed successfully in case there were observers available in the shadow area. %Fig. \ref{Fig: occ-Himalia} shows the maps made for 4 different offset corrections as explained in Sec. \ref{Sec: testes}. Table \ref{Tab: comparison-Himalia} shows numerically the difference between the maps.
%A similar test was made for an occultation of Elara in March 30, 2015. %The difference between the maps in this case are listed in Table \ref{Tab: comparison-Elara}.
The results show satisfying small offsets with respect to the local of the prediction. %We can expect similar results for other satellites and our method should allow successful stellar occultation.

Continuous observations of the satellites are recommended and fitting of our dynamical model to those observations are expected to reduce the respective STE ephemeris errors. The first version of the GAIA catalogue is to be released up to the end of 2016 and will improve the position error of the stars to the 1-5 mas level. It will allow for the discovery of ocultations by fainter stars not present in the UCAC4 catalogue. % Consequently, the astrometry of the satellites with the GAIA catalogue will improve the position errors allowing for better predictions and a higher probabilities of observing stellar occultations by irregular satellites.
The release of the GAIA catalogue should have a positive impact on both the astrometric precision of occulted stars and the reduction of astrometric positions of the satellites. As a result, prediction of stellar occultations by irregular satellites shall increase in number as well as in success.

%We managed a large database with FITS images acquired by 5 telescopes in 3 sites between 1992 and 2014. From that, we identified 8466 observations of irregular satellites, from which we managed to obtain 6523 suitable astrometric positions, giving a total of 3666 positions for 12 satellites of Jupiter, 1920 positions for 4 satellites of Saturn, 35 positions for Sycorax (Uranus) and 902 positions for Nereid (Neptune).

%The positions of all the objects were determined using the PRAIA package. The package was suited to cope with the huge amount of observations and the task of identifying the satellites within the database. PRAIA tasks were also useful to deal with the missing or incorrect coordinate and time stamps present mostly in the old observations.The UCAC4 was used as the reference frame. Based in the comparisons with ephemeris, we estimate that the position errors are about 60 mas to 80 mas depending on the satellite brightness.

%For some satellites the number of positions obtained in this work is comparable to the number used in the numerical integration of orbits by the JPL \citep{Jacobson2012} (see Table \ref{Tab: comparison-horizons}). For instance, the amount of new positions for Himalia is about 70\% of the number used in the numerical integation of orbits by JPL. Systematic errors in the ephemeris were found for at least some satellites (Ananke, Carme, Elara and Pasiphae). In the case of Carme, we evidenced an error in the orbital inclination (see Fig. \ref{Fig: carme_anom}).

%The positions derived in this work can be used in new orbital numerical integrations, generating more precise ephemerides. Stellar occultations by irregular satellites could then be better predicted. Based in this work, our group has already computed occultation predictions for the 8 major irregular satellites of Jupiter. These predictions will be published in a forthcoming paper.