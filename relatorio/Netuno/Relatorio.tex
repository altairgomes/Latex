	\documentclass[12pt,a4paper]{report}
\usepackage[brazilian, english]{babel}
\usepackage[utf8]{inputenc}
\usepackage[T1]{fontenc}
\usepackage{amsmath,amsthm,amsfonts,amssymb,textcomp}
%\usepackage{latexsym}
\usepackage{graphicx}
\graphicspath{{figuras}}
\usepackage{subfigure}
\usepackage{float}
\usepackage{longtable}
\usepackage{color}
\usepackage{epstopdf}
\usepackage{pdflscape}
\usepackage[breaklinks=true]{hyperref}
\usepackage[comma,authoryear]{natbib}
\usepackage[nonumberlist]{glossaries}
\usepackage{arydshln}
\usepackage{footnote}
\usepackage{longtable}
\usepackage[small,bf,singlelinecheck=off]{caption}
\usepackage[left=3cm,right=2cm,top=3cm,bottom=2cm]{geometry}
%\usepackage[alf]{abntcite}

%%% newcommand %%%%%%%%%%%%%%%%%
\newcommand{\PE}{Perkin-Elmer }
\newcommand{\BC}{Boller \& Chivens }

\newcommand{\degr}{\ensuremath{^{\circ}}}%                    % degree symbol:  °
\newcommand{\arcmin}{\ensuremath{^{\prime}}}%                    % degree symbol:  °
\newcommand{\arcsec}{\ensuremath{^{\prime\prime}}}%                    % degree symbol:  °
\newcommand{\fs}{\mbox{\ensuremath{.\!\!^s}}}
\newcommand{\farcm}{\mbox{\ensuremath{.\mkern-4mu^\prime}}}%    % fractional arcminute symbol: 0.'0
\newcommand{\farcs}{\mbox{\ensuremath{.\!\!^{\prime\prime}}}}%  % fractional arcsecond symbol: 0.''0
\newcommand{\fdg}{\mbox{\ensuremath{.\!\!^\circ}}}%             % fractional degree symbol:     0.°0



%\makeglossaries

\makeatletter
\renewcommand\chapter{\thispagestyle{plain}
                \global\@topnum\z@
                \@afterindentfalse
                \secdef\@chapter\@schapter}
\makeatother


\makeatletter
\renewcommand{\@makechapterhead}[1]{%
\vspace*{50 pt}%
{\setlength{\parindent}{0pt} \raggedright \normalfont
\bfseries\Huge\thechapter.\ #1
\par\nobreak\vspace{40 pt}}}
\makeatother

%\newglossaryentry{Offset}{name={Offset}, description={Diferença entre a posição obtida pela redução da observação e a posição dada pela efeméride}}
\newglossaryentry{OPD}{name={OPD}, description={Observatório do Pico dos Dias - Brasópolis, MG}}
\newglossaryentry{LNA}{name={LNA}, description={Laboratório Nacional de Astrofísica - Itajubá, MG}}
\newglossaryentry{OHP}{name={OHP}, description={Observatoire Haute Provence - Saint-Michel-l'Observatoire, França}}
\newglossaryentry{RA}{name={RA}, description={Sigla para Ascensão Reta ($ \alpha $)}}
\newglossaryentry{DEC}{name={DEC}, description={Sigla para Declinação ($ \delta $)}}
\newglossaryentry{Anomalia Verdadeira}{name={Anomalia Verdadeira}, description={Ângulo formado entre o Periastro e a posição instantânea do objeto na órbita centrado no planeta e contada na direção do movimento orbital}}
\author{Altair Ramos Gomes Júnior}
\title{Astrometry of the Neptune-Triton System}
\begin{document}

\maketitle

\pagestyle{headings}

\section*{Introduction}

In this report I present the preliminary results of the astrometric reductions of the images from the Observatório do Pico dos Dias (OPD) in Brazil. The aim is to obtain precise positions for the Neptune - Triton system and to investigate the orbit of Triton around Neptune. Eventually, also investigate the orbit of Neptune around the Sun. The telescopes used were the \PE (160) with a diameter of 1.6m, the \BC (IAG) with a diameter of 0.6m, and the Zeiss telescope with a diameter of 0.6m.

The observations were carried out since 1992 when a CCD big enough was installed in the OPD. The planet and satellite have been constantly observed, and still are, by our group. There were many CCDs (IKON, IXON, CCD101, CCD106, ...) and many filters (V, R, I, No Filter, ...) utilized.

There was more than 5000 images from June 1992 to September 2015. Many of the oldest images had no coordinates in header or they were wrong. Many nights had two exposure sets. The first one with low exposure times so Neptune was not saturated, but there were few reference stars in the field. The second one with higher exposure time so Triton was brighter and had more reference stars than with the the previous exposure, but the image of Neptune were saturated.

In Table \ref{Tab:dados} it is summarized the final number of (only non-saturated) images for Neptune (short-exposure observations) and Triton (all observations) for the 3 telescopes. It is also shown the number of positions where non-saturated-Neptune and Triton are in the same image (short-exposure observations for precision premium).

\begin{table*}[h]
\centering
\caption{Number of positions by object by telescope}
\label{Tab:dados}
\begin{tabular}{|c|c|c|c|}
\hline 
Telescope & Neptune & Triton & Matches \\ 
\hline
%160 & 782 & 1341 & 768 \\ 
160 & 735 & 1251 & 682 \\
\hline
%IAG & 3162 & 3645 & 2909 \\
IAG & 2795 & 3341 & 2459 \\ 
\hline 
%Zeiss & 354 & 479 & 341 \\
Zeiss & 292 & 463 & 280 \\ 
\hline 
%Total & 4298 & 5465 & 4018 \\
Total & 3822 & 5055 & 3421 \\ 
\hline 
\end{tabular}
\\Number of positions identified of non-saturated-Neptune and Triton by telescope. Matches: Number of positions where non-saturaded-Neptune and Triton were identified automatically in the same image. This is the final number after all the process described in this report.
\end{table*}

Fig. \ref{Fig:pos-dist} shows the distribution of positions where Neptune and Triton are identified in the same image (short-exposure observations) over the years. Table \ref{Tab:filters} summarizes the distribution of positions with Neptune and Triton in the same image by filter for each telescope.

\begin{figure}
\includegraphics[width=16.0cm]{pos-distribution.png} 
\caption{Distribution of positions with Neptune and Triton in the same image (short-exposure observations) by year.}
\label{Fig:pos-dist}
\end{figure}


\begin{table*}
\centering
\begin{tabular}{|l|c|c|c|c|c|c|}
\hline
Telescope & Clear & B & V & R & I & Metano\\
\hline
160 & 569 & 4 & 5 & 5 & 86 & 13 \\
IAG & 919 & 21 & 126 & 243 & 1032 & 118 \\
Zeiss & 218 & - & - & - & 62 & - \\
\hline
\end{tabular}
\caption{Number of positions with Neptune and Triton in the same image by filter for each telescope.}
\label{Tab:filters}
\end{table*}

\section*{Reduction}

The images were reduced using PRAIA, developed by Marcelo Assafin. To avoid the missing or wrong field-center coordinates I used the coordinates of the ephemeris as input. This way PRAIA could identify reference stars in the images. The reference catalogue used was UCAC4. The ephemeris used to identify Neptune and Triton in the images was DE430+NEP081. The positions where the image of Neptune was saturated and where there were less than 5 reference stars were removed of the results.

In Table \ref{Tab:erros} it is presented the mean errors in X and Y of the bidimensional circular Gaussian used to fit the PSF of the objects and the mean value of the dispersion of the offsets by night.

\begin{longtable}{|l|c|c|c|}
\caption{Table of errors of the reduction. Gaussian error stands for the error in X and Y of the bidimensional circular Gaussian PSF used in the (x,y) fits. Mean offset errors is the average dispersion of the positions of each night.\label{Tab:erros}}\\
\hline
Telescope/Satellite &  \multicolumn{1}{|c|}{Gaussian error} &   \multicolumn{2}{|c|}{Mean offet errors} \\
%\hline
 &  (mas) & RA (mas) & DEC (mas) \\
\hline
\endfirsthead
\multicolumn{4}{c}%
{\tablename\ \thetable\ -- \textit{Continued from previous page}} \\
\hline
Telescope/Satellite & Gaussian & RA & DEC \\
\hline
\endhead
\hline \multicolumn{4}{r}{\textit{Continued on next page}} \\
\endfoot
\hline
\endlastfoot
160/Neptune & 8$\pm$4 & 51 & 39 \\
160/Triton & 14$\pm$8 & 35 & 38 \\
IAG/Neptune & 9$\pm$7 & 63 & 58 \\
IAG/Triton & 20$\pm$14 & 52 & 53 \\
Zeiss/Neptune & 9$\pm$6 & 49 & 57 \\
Zeiss/Triton & 25$\pm$13 & 40 & 51 \\
\hline
\end{longtable}

We applied the digital coronagraphy technique to test if the scattered light of Neptune would influence in the Triton's photocenter. We found no variations larger than 1 mas. We thus concluded that we do not need digital coronagraphy to improve the positions, as was the case of Uranus and its satellite Miranda.

\newpage
\section*{Chromatic Refraction Test}

Table \ref{Tab:colors} shows the colors for Triton \citep{Pascu2006} and Neptune \cite{Schmude2016}. Their colors are very different. Neptune is much bluer than Triton. So it is expected that their positions have influence of chromatic refraction (CR) with different intensities with respect to the reference stars. The apparent position of Neptune, which is more blue than Triton, would be more shifted towards the zenith than the Triton's position. There may also be noted that in 1992 Neptune had just exited the galactic plane, so the reference stars were redder due to dust.

\begin{table*}[h]
\centering
\begin{tabular}{|l|c|c|c|c|c|}
\hline
Object & U-B & B-V & V-R & R-I & V-I\\
\hline
Triton (leading side) & & +0.696$\pm$0.009 & & & +0.766$\pm$0.006 \\
Triton (trailing side) & & +0.699$\pm$0.006 & & & +0.776$\pm$0.007 \\
Neptune & +0.14 & +0.39 & -0.29 & -1.05 & -1.34*\\
\hline
\end{tabular}
\caption{Colors of Triton and Neptune. Leading side is the hemisphere of Triton that is in the direction of its movement. Trailing side is the opposite hemisphere. \\ *calculated from V-R and R-I colors.}
\label{Tab:colors}
\end{table*}

\citep{Pascu2006} data also support a secular "blueing" on Triton observed since 1954. They also evidence a reddening episode that happened in 1997 where the B-V color of Triton was bigger than 0.9. A similar reddening was also identified in 1979 (Fig. \ref{Fig:Pascu}). The authors state that a possible cause of this event is an increase in the activity of geysers.

\begin{figure}[h]
\centering
\includegraphics[scale=0.7]{Pascu.png}
\caption{Figure extracted from \cite{Pascu2006}. \label{Fig:Pascu}}
\end{figure}

For Neptune, \cite{Schmude2016} showed a secular brightening in the B-, V-, R- and I-bands (Fig. \ref{Fig:netuno-secular}) from observations since 1954. They also identified, from Hubble observations, that Neptune has a variation of about 1 magnitude in the I-band over some hours caused by the presence of bright clouds on its atmosphere (Fig. \ref{Fig:netuno-nuvem}). All these circumstances can difficult the estimative of chromatic refraction parameters for Neptune and Triton.

\begin{figure}[H]
\centering
\includegraphics[scale=1]{Neptune_secular.png}
\caption{Secular brightening of Neptune in B-, V-, R- and I-bands \citep{Schmude2016}. \label{Fig:netuno-secular}}
\end{figure}

\begin{figure}[H]
\centering
\includegraphics[scale=1]{netuno-day.png}
\caption{HST images of Neptune at $\lambda=8450 \mbox{\normalfont\AA}$ which show a variation of the I-magnitude caused by bright clouds on the atmosphere \citep{Schmude2016}.\label{Fig:netuno-nuvem}}
\end{figure}



To test the effects of chromatic refraction I used the method of \cite{Benedetti-Rossi2014} on all nights with observations distributed over more than 1.5h of hour angle. I used the equation \:
\begin{equation}
\Delta [\alpha, \delta] = V_{\alpha,\delta} (\phi,\delta, H) \cdot \Delta B,
\label{Eq:refraction}
\end{equation}
where
\begin{equation}
V_\alpha (\phi, \delta, H) = \frac{\sec^2 \delta . \sin H}{\tan \delta . \tan \phi + \cos H} \qquad and \qquad V_\delta (\phi, \delta, H) = \frac{\tan \phi - \tan \delta . \cos H}{\tan \delta . \tan \phi + \cos H}
\label{Eq:refraction}
\end{equation}
to model the chromatic refraction of the nights. $\Delta [\alpha, \delta]$ is the position offset for each coordinate ($\alpha, \delta$), $V_{\alpha,\delta} (\phi,\delta, H)$ is the first part of refraction which is due to the position of the observed objects and is a function of the latitude of the site ($\phi$), of the object’s declination ($\delta$), and of the hour angle ($H$) and $\Delta B$ is the the second part: the differential chromatic refraction which is due to the atmospheric conditions and the wavelength ($\lambda$) of the object and of the stars in the field. This equation is available in \cite{Benedetti-Rossi2014} where it was applied for observations of Pluto. Notice that positive values for $\Delta B$ indicates that the body's color is bluer than the average color of the reference stars used in the (RA, DEC) reductions.

The model is applied to the offsets in $\alpha$ and the chromatic parameter $\Delta B$ is obtained. It allows to obtain the zero point of the CR in right ascension since the CR in the meridian must be null. This parameter is then used to correct the offsets in $\delta$, where the effect of CR is much smaller because the declination of Neptune is close to the latitude of the observation site ($\delta = -21$\degr in 1992 up to $\delta = -9$\degr in 2015, OPD: $\phi = -22.5$\degr).

Fig. \ref{Fig:refraction-sample-neptune} shows the offsets for a sample night observed with the Perkin-Elmer telescope. In blue are the offsets before the correction and in green after correction. It is possible to see the increase in the offsets before the correction over time (blue).

For the nights with observations distributed in a smaller hour angle, the correction was made following the conditions:

\begin{itemize}
\item If the night only has observations between -1h and 1h of hour angle, no correction is made.
\item If there is another night with $\Delta B$ calculated observed with the same filter and same telescope within at most 3 days apart, the $\Delta B$ from that night is used in the CR correction.
\item If there is no close night with $\Delta B$ calculated, it is used the mean $\Delta B$ calculated for nights observed with the same filter and same telescope for the correction of CR.
\item Other situations, no CR correction is made.
\end{itemize}

Figs. \ref{Fig:refraction-net-160}-\ref{Fig:refraction-tri-iag} show the distributions of the offsets in RA and DEC before and after the elimination of chromatic refraction (Neptune-160, Triton-160, Neptune-IAG, Triton-IAG, respectively) for all the nights. Significant improvements were achieved in right ascension as expected, while in declination they are much smaller. As expected, more significant improvements were achieved in right ascension than in declination, because the declination of Neptune is close to the latitude of the OPD site ($\delta = -21$\degr in 1992 up to $\delta = -9$\degr in 2015, OPD: $\phi = -22.5$\degr).
We see that the Neptune's offsets are much more improved than for Triton's, as expected since Neptune is much bluer than the satellite. %And we see a significant improvement for Neptune positions, also expected because of its much bluer color. % looked for nights with many observations distributed over the night. I selected five of them (\PE: 1997-05-31 and 1997-06-01; \BC: 2003-07-27, 2004-08-22 and 2011-09-25) with more than 4 hours between the first and last observations. In the nights of \PE and in the 2003 night of \BC it was used no filter (clear). In the night of 2004 it was used a filter called "Dark" which we suppose it means no filter, however we are no sure so I classified it as unknown. In the night of 2011 it was used a I filter.% In both nights there was only one exposure time used.

%\textcolor{red}{colocar que era esperado que pra Netuno a correção é maior do que para Tritão porque ele é mais azulado}

%In Fig \ref{Fig:refraction} it is plotted the difference in the positions of Triton and Neptune compared to the difference of their ephemeris in RA and DEC for the 5 nights.

\begin{figure}[H]
\centering
\includegraphics[height=11.0cm]{Netuno_160_1993-08-20.png} 
\caption{Offsets in right ascension before (red) and after (blue) CR correction for the night of August 20, 1993 observed with the Perkin-Elmer telescope.}
\label{Fig:refraction-sample-neptune}
\end{figure}

\begin{figure}[H]
\centering
\includegraphics[height=11.0cm]{dist_Netuno_160.png} 
\caption{Distribution of the offsets of Neptune observed in 160.}
\label{Fig:refraction-net-160}
\end{figure}
\begin{figure}[H]
\centering
\includegraphics[height=11.0cm]{dist_Triton_160.png} 
\caption{Distribution of the offsets of Triton observed in 160.}
\label{Fig:refraction-tri-160}
\end{figure}
\begin{figure}[H]
\centering
\includegraphics[height=11.0cm]{dist_Netuno_IAG.png} 
\caption{Distribution of the offsets of Neptune observed in IAG.}
\label{Fig:refraction-net-iag}
\end{figure}
\begin{figure}[H]
\centering
\includegraphics[height=11.0cm]{dist_Triton_IAG.png} 
\caption{Distribution of the offsets of Triton observed in IAG.}
\label{Fig:refraction-tri-iag}
\end{figure}

\begin{landscape}

Table \ref{Tab:parameters} shows the nights used to calculate the CR parameter (nights with $\Delta H$ > 1.5). It is possible to see that $\Delta B$ has higher values for Neptune than for Triton, as expected from the much bluer color of the planet with regard to the satellite's color. Due to the low number of reference stars, the mean color of the stars may vary significantly among the nights. This causes the high variation in the $\Delta B$ parameter seen in column 4 of the table, both for Neptune and Triton. 

\begin{longtable}{|l|c|c|c|c|c|c|c|c|c|c|c|c|}
\caption{Obtained parameters and offsets from adjustments. Only nights with $\Delta H > 1.5h$. Also shown the filter, the variation in hour angle ($\Delta H$), the parameter obtained ($\Delta B$), number of images (Nimg), mean number of reference stars (Nstars), mean offsets before and after correction and mean difference between the non corrected and corrected offsets.\label{Tab:parameters}}\\
\hline
\multicolumn{12}{|c|}{Neptune-160}\\
Date & Filter & $\Delta H$ & $\Delta B$ & Nimg & Nstars & RA no corr & DEC no corr & RA corr & DEC corr & $\Delta$RA & $\Delta$DEC \\
%\hline
\hline
\endfirsthead
\multicolumn{12}{c}%
{\tablename\ \thetable\ -- \textit{Continued from previous page}} \\
\hline
Date & Filter & $\Delta H$ & $\Delta B$ & Nimg & Nstars & RA no corr & DEC no corr & RA corr & DEC corr & $\Delta$RA & $\Delta$DEC \\
\hline
\endhead
\hline \multicolumn{12}{r}{\textit{Continued on next page}} \\
\endfoot
\hline
\endlastfoot
%\begin{table*}
%\begin{tabular}{|l|c|c|c|c|c|c|c|c|c|c|}
%\hline
%Date & Filter & $\Delta H$ & $\Delta B$ & Nimg & Nstars & RA no corr & DEC no corr & RA corr & DEC corr \\
\hline
1992-06-09 & Clear & 1.57 & +0.29$\pm$0.03 &  13 &  11 &  190$\pm$ 54 &   25$\pm$ 76 &  -43$\pm$ 15 &   65$\pm$ 68 &  233$\pm$ 51 &  -41$\pm$ 12 \\ 
1992-07-19 & Clear & 1.63 & +0.20$\pm$0.04 &  17 &  24 &   50$\pm$ 46 &  117$\pm$ 37 &    1$\pm$ 26 &  125$\pm$ 36 &   49$\pm$ 37 &   -7$\pm$  4 \\ 
1993-06-24 & Clear & 1.61 & -0.07$\pm$0.04 &  10 &  17 &  -17$\pm$ 26 &  -38$\pm$ 48 &  -21$\pm$ 22 &  -40$\pm$ 48 &    4$\pm$ 14 &    2$\pm$  0 \\ 
1993-06-25 & Clear & 2.63 & +0.03$\pm$0.02 &  12 &   8 &  -37$\pm$ 26 &   70$\pm$ 75 &  -40$\pm$ 24 &   72$\pm$ 74 &    2$\pm$ 11 &   -1$\pm$  1 \\ 
1993-08-20 & Clear & 3.35 & +0.20$\pm$0.02 &  31 &  29 &   15$\pm$ 76 &  -83$\pm$ 43 &   -7$\pm$ 34 &  -75$\pm$ 44 &   21$\pm$ 68 &   -9$\pm$  4 \\ 
1993-08-22 & Clear & 2.87 & +0.18$\pm$0.02 &  35 &   9 &  -47$\pm$ 63 &  -74$\pm$ 44 &  -15$\pm$ 38 &  -67$\pm$ 46 &  -32$\pm$ 50 &   -7$\pm$  4 \\ 
1996-06-22 & CLEAR & 1.68 & +0.24$\pm$0.02 &  19 &  16 &  -25$\pm$ 51 &  -32$\pm$ 35 &  -48$\pm$ 18 &  -20$\pm$ 35 &   23$\pm$ 47 &  -12$\pm$  2 \\ 
1996-10-02 & Clear & 1.97 & +0.31$\pm$0.07 &  16 &   6 &  252$\pm$192 &  -60$\pm$ 73 &  -59$\pm$126 &   10$\pm$ 79 &  311$\pm$145 &  -70$\pm$ 46 \\ 
1997-06-01 & Clear & 4.89 & +0.35$\pm$0.03 &  91 &  11 & -155$\pm$189 & -112$\pm$ 36 & -120$\pm$107 &  -81$\pm$ 40 &  -35$\pm$156 &  -31$\pm$ 11 \\ 
1997-06-02 & Clear & 5.24 & +0.22$\pm$0.01 &  60 &  10 & -113$\pm$125 &  -84$\pm$ 36 &  -73$\pm$ 48 &  -61$\pm$ 33 &  -40$\pm$115 &  -23$\pm$ 12 \\ 
1998-06-06 & Clear & 2.51 & +0.38$\pm$0.04 &  35 &  11 & -140$\pm$101 &  -66$\pm$ 42 &  -13$\pm$ 56 &  -33$\pm$ 41 & -127$\pm$ 83 &  -33$\pm$ 11 \\ 
1998-09-03 & Clear & 1.52 & +0.30$\pm$0.03 &  20 &  10 &  -86$\pm$ 63 &  -75$\pm$ 49 &   20$\pm$ 26 &  -51$\pm$ 47 & -106$\pm$ 57 &  -24$\pm$  8 \\ 
\hline
\multicolumn{12}{|c|}{Triton-160}\\
Date & Filter & $\Delta H$ & $\Delta B$ & Nimg & Nstars & RA no corr & DEC no corr & RA corr & DEC corr & $\Delta$RA & $\Delta$DEC \\
\hline
1992-06-09 & Clear & 1.57 & +0.01$\pm$0.03 &  16 &  16 &   12$\pm$ 20 &   19$\pm$ 79 &    3$\pm$ 20 &   20$\pm$ 79 &   10$\pm$  2 &   -2$\pm$  1 \\ 
1992-07-19 & Clear & 1.89 & +0.07$\pm$0.04 &  21 &  25 &   -3$\pm$ 33 &   15$\pm$ 52 &  -21$\pm$ 30 &   17$\pm$ 52 &   18$\pm$ 14 &   -3$\pm$  1 \\ 
1993-06-24 & Clear & 1.61 & +0.01$\pm$0.04 &  15 &  13 &    8$\pm$ 23 &   15$\pm$ 55 &    9$\pm$ 23 &   16$\pm$ 55 &   -1$\pm$  1 &   -0$\pm$  0 \\ 
1993-06-25 & Clear & 2.90 & -0.09$\pm$0.04 &  20 &  12 &  -28$\pm$ 60 &   36$\pm$ 79 &  -14$\pm$ 52 &   32$\pm$ 80 &  -14$\pm$ 28 &    4$\pm$  2 \\ 
1993-08-20 & Clear & 3.35 & -0.01$\pm$0.02 &  30 &  27 &   -8$\pm$ 29 &  -25$\pm$ 62 &   -6$\pm$ 29 &  -25$\pm$ 62 &   -2$\pm$  5 &    1$\pm$  0 \\ 
1993-08-22 & Clear & 3.12 & +0.05$\pm$0.01 &  43 &  13 &   -2$\pm$ 28 &   -9$\pm$ 43 &    4$\pm$ 24 &   -7$\pm$ 43 &   -7$\pm$ 14 &   -2$\pm$  1 \\ 
1994-09-22 & Clear & 1.94 & -0.12$\pm$0.08 &  13 &  12 &  -33$\pm$ 53 &   30$\pm$ 75 &   48$\pm$ 49 &   16$\pm$ 71 &  -81$\pm$ 22 &   14$\pm$  6 \\ 
1994-09-22 & Clear & 1.55 & +0.00$\pm$0.04 &  15 &  17 &  -53$\pm$ 21 &   24$\pm$ 82 &  -54$\pm$ 21 &   24$\pm$ 82 &    0$\pm$  1 &   -0$\pm$  0 \\ 
1995-08-07 & Clear & 3.02 & +0.02$\pm$0.02 &  11 &  21 &  -10$\pm$ 14 &  -40$\pm$ 30 &  -10$\pm$ 13 &  -39$\pm$ 30 &    0$\pm$  5 &   -1$\pm$  0 \\ 
1996-06-22 & CLEAR & 2.28 & -0.08$\pm$0.02 &  32 &  15 &  -87$\pm$ 24 &   -9$\pm$ 29 &  -77$\pm$ 19 &  -13$\pm$ 28 &  -10$\pm$ 15 &    4$\pm$  1 \\ 
1996-08-22 & Clear & 1.56 & +0.01$\pm$0.02 &  29 &  13 &   43$\pm$ 30 &  -41$\pm$ 44 &   37$\pm$ 30 &  -40$\pm$ 44 &    5$\pm$  2 &   -1$\pm$  0 \\ 
1996-08-24 & Clear & 1.99 & -0.03$\pm$0.04 &  10 &  11 &   46$\pm$ 15 &  -66$\pm$ 23 &   52$\pm$ 15 &  -67$\pm$ 23 &   -5$\pm$  4 &    1$\pm$  0 \\ 
1996-10-02 & Clear & 1.97 & +0.04$\pm$0.07 &  16 &   6 &   73$\pm$121 &    1$\pm$ 76 &   38$\pm$120 &    9$\pm$ 77 &   35$\pm$ 16 &   -8$\pm$  5 \\ 
1997-06-01 & Clear & 4.89 & +0.11$\pm$0.01 & 101 &  11 &  -76$\pm$ 71 & -107$\pm$ 53 &  -68$\pm$ 51 &  -98$\pm$ 54 &   -8$\pm$ 49 &  -10$\pm$  3 \\ 
1997-06-02 & Clear & 5.41 & +0.02$\pm$0.02 &  81 &  10 &  -18$\pm$ 86 &  -56$\pm$ 30 &  -17$\pm$ 85 &  -54$\pm$ 30 &   -1$\pm$ 10 &   -2$\pm$  1 \\ 
1997-08-11 & Clear & 3.08 & +0.12$\pm$0.02 &  33 &  11 &  -30$\pm$ 46 &  -23$\pm$ 27 &    2$\pm$ 28 &  -14$\pm$ 28 &  -32$\pm$ 37 &   -9$\pm$  3 \\ 
1997-08-13 & Clear & 1.67 & +0.04$\pm$0.03 &  19 &   6 &   44$\pm$ 30 &   -4$\pm$ 16 &   60$\pm$ 29 &   -0$\pm$ 16 &  -16$\pm$  8 &   -3$\pm$  1 \\ 
1998-06-06 & Clear & 2.84 & +0.13$\pm$0.04 &  42 &  11 &  -46$\pm$ 62 &  -55$\pm$ 33 &   -5$\pm$ 53 &  -44$\pm$ 33 &  -42$\pm$ 31 &  -11$\pm$  4 \\ 
1998-09-03 & Clear & 1.52 & -0.04$\pm$0.04 &  20 &  10 &    3$\pm$ 36 &  -48$\pm$ 54 &  -11$\pm$ 35 &  -50$\pm$ 54 &   13$\pm$  7 &    3$\pm$  1 \\ 
1999-06-06 & Clear & 1.58 & +0.23$\pm$0.04 &  27 &  14 &   19$\pm$ 47 &  -63$\pm$ 43 &   91$\pm$ 32 &  -43$\pm$ 40 &  -71$\pm$ 34 &  -20$\pm$  4 \\ 
1999-08-22 & Clear & 3.06 & +0.03$\pm$0.02 &  30 &  15 &  -32$\pm$ 37 &   -7$\pm$ 25 &  -45$\pm$ 35 &   -3$\pm$ 26 &   13$\pm$ 11 &   -3$\pm$  2 \\ 
\hline
\multicolumn{12}{|c|}{Neptune-IAG}\\
Date & Filter & $\Delta H$ & $\Delta B$ & Nimg & Nstars & RA no corr & DEC no corr & RA corr & DEC corr & $\Delta$RA & $\Delta$DEC \\
\hline
2001-08-26 & B     & 2.11 & +0.15$\pm$0.03 &  44 &  15 &   89$\pm$ 55 &  -41$\pm$ 57 &   14$\pm$ 41 &  -22$\pm$ 57 &   75$\pm$ 36 &  -19$\pm$  6 \\ 
2002-07-15 & Clear & 6.11 & +0.18$\pm$0.00 &  57 &  17 &   51$\pm$163 &   85$\pm$ 90 &  -36$\pm$ 31 &  131$\pm$ 76 &   87$\pm$160 &  -46$\pm$ 20 \\ 
2002-07-18 & CLEAR & 3.86 & +0.21$\pm$0.02 &  30 &  13 & -100$\pm$115 & -140$\pm$ 61 &  -50$\pm$ 61 & -111$\pm$ 61 &  -50$\pm$ 98 &  -29$\pm$  3 \\ 
2003-07-22 & Clear & 2.38 & +0.33$\pm$0.04 &  20 &  15 &  174$\pm$137 &  -75$\pm$ 80 &  -32$\pm$ 57 &  -13$\pm$ 74 &  206$\pm$125 &  -62$\pm$ 27 \\ 
2003-07-23 & Clear & 4.08 & +0.02$\pm$0.01 &  39 &  16 &  -81$\pm$ 36 &   13$\pm$ 55 &  -78$\pm$ 34 &   17$\pm$ 55 &   -4$\pm$ 12 &   -3$\pm$  1 \\ 
2003-07-25 & Clear & 6.19 & -0.01$\pm$0.02 &  21 &   9 & -107$\pm$ 74 &    9$\pm$ 62 & -106$\pm$ 74 &    7$\pm$ 62 &   -1$\pm$  8 &    2$\pm$  1 \\ 
2003-07-26 & Clear & 6.97 & -0.01$\pm$0.08 &  17 &  10 & -132$\pm$211 &    8$\pm$123 & -129$\pm$211 &    7$\pm$124 &   -3$\pm$  4 &    1$\pm$  1 \\ 
2003-07-27 & Clear & 1.54 & +0.06$\pm$0.06 &  26 &  14 &  -21$\pm$104 &    4$\pm$ 84 &  -90$\pm$102 &   24$\pm$ 84 &   70$\pm$ 20 &  -20$\pm$  6 \\ 
2003-07-28 & Clear & 7.98 & +0.01$\pm$0.03 &  43 &  12 &  -37$\pm$144 &   21$\pm$162 &  -44$\pm$144 &   24$\pm$162 &    7$\pm$  4 &   -2$\pm$  1 \\ 
2003-08-20 & CLEAR & 4.03 & +0.26$\pm$0.04 &  30 &  17 &   95$\pm$154 &  -72$\pm$ 66 &   18$\pm$ 98 &  -35$\pm$ 60 &   76$\pm$118 &  -37$\pm$ 16 \\ 
2003-10-14 & V     & 2.33 & +0.02$\pm$0.03 &   8 &  30 &   20$\pm$ 25 &   16$\pm$ 31 &   16$\pm$ 25 &   18$\pm$ 31 &    5$\pm$  5 &   -2$\pm$  1 \\ 
2004-08-05 & V     & 2.21 & -0.03$\pm$0.12 &   5 &   6 &  -53$\pm$ 67 &  -61$\pm$ 25 &  -35$\pm$ 67 &  -66$\pm$ 26 &  -18$\pm$  9 &    5$\pm$  2 \\ 
2004-08-07 & V     & 4.31 & +0.21$\pm$0.33 &   6 &  11 &  128$\pm$333 &  -40$\pm$210 &  110$\pm$317 &   -9$\pm$212 &   18$\pm$102 &  -31$\pm$  8 \\ 
2004-08-21 & Clear & 4.09 & +0.08$\pm$0.02 &  30 &  21 &  -32$\pm$ 57 &  -78$\pm$ 55 &  -56$\pm$ 44 &  -66$\pm$ 57 &   24$\pm$ 36 &  -12$\pm$  4 \\ 
2004-08-21 & Clear & 2.57 & +0.05$\pm$0.02 &  30 &  21 &  -42$\pm$ 28 &  -86$\pm$ 41 &  -32$\pm$ 26 &  -81$\pm$ 41 &  -10$\pm$ 12 &   -6$\pm$  1 \\ 
2004-08-23 & Clear & 5.20 & +0.06$\pm$0.01 &  70 &  18 &   10$\pm$ 59 &  -88$\pm$ 52 &  -29$\pm$ 42 &  -73$\pm$ 50 &   40$\pm$ 42 &  -14$\pm$ 10 \\ 
2004-08-24 & Clear & 3.94 & +0.06$\pm$0.01 &  40 &  18 &   -2$\pm$ 34 &  -83$\pm$ 66 &  -23$\pm$ 26 &  -74$\pm$ 65 &   20$\pm$ 23 &   -8$\pm$  4 \\ 
2004-09-24 & R     & 3.08 & +0.27$\pm$0.05 &  35 &  13 &  157$\pm$183 &   12$\pm$136 &   -0$\pm$133 &   63$\pm$126 &  157$\pm$126 &  -51$\pm$ 28 \\ 
2004-09-25 & Clear & 3.75 & +0.26$\pm$0.03 &  40 &  14 &  201$\pm$164 &    2$\pm$ 93 &   37$\pm$106 &   54$\pm$ 86 &  165$\pm$126 &  -52$\pm$ 32 \\ 
2005-09-24 & V     & 2.78 & +0.05$\pm$0.01 &  88 &  14 &    2$\pm$ 47 & -101$\pm$ 61 &  -52$\pm$ 44 &  -85$\pm$ 60 &   54$\pm$ 17 &  -16$\pm$  4 \\ 
2006-06-08 & Clear & 2.68 & +0.32$\pm$0.04 &  95 &  24 &  -91$\pm$144 &  -83$\pm$ 93 &   14$\pm$115 &  -30$\pm$ 93 & -105$\pm$ 87 &  -53$\pm$  8 \\ 
2011-09-26 & I     & 4.02 & +0.05$\pm$0.00 & 250 &  18 &   76$\pm$ 44 &  -67$\pm$ 48 &   38$\pm$ 36 &  -52$\pm$ 45 &   38$\pm$ 25 &  -15$\pm$  7 \\ 
2012-10-19 & R     & 1.99 & +0.13$\pm$0.03 & 119 &   8 &   41$\pm$100 & -167$\pm$137 &  -50$\pm$ 92 & -130$\pm$138 &   90$\pm$ 38 &  -37$\pm$  8 \\ 
\hline
\multicolumn{12}{|c|}{Triton-IAG}\\
Date & Filter & $\Delta H$ & $\Delta B$ & Nimg & Nstars & RA no corr & DEC no corr & RA corr & DEC corr & $\Delta$RA & $\Delta$DEC \\
\hline
2001-08-26 & B     & 2.11 & +0.03$\pm$0.03 &  24 &  15 &    1$\pm$ 40 &  -53$\pm$ 46 &  -13$\pm$ 39 &  -50$\pm$ 46 &   14$\pm$  7 &   -4$\pm$  1 \\ 
2002-07-15 & Clear & 6.11 & +0.01$\pm$0.00 &  55 &  17 &   14$\pm$ 27 &  -29$\pm$ 45 &    7$\pm$ 24 &  -26$\pm$ 45 &    7$\pm$ 12 &   -3$\pm$  2 \\ 
2002-07-18 & CLEAR & 3.86 & -0.01$\pm$0.03 &  31 &  13 &   25$\pm$ 74 &  -68$\pm$111 &   23$\pm$ 74 &  -69$\pm$111 &    2$\pm$  4 &    1$\pm$  0 \\ 
2003-07-22 & Clear & 2.55 & +0.08$\pm$0.02 &  31 &  17 &   -8$\pm$ 57 &   -7$\pm$ 53 &  -58$\pm$ 49 &    8$\pm$ 54 &   50$\pm$ 29 &  -15$\pm$  7 \\ 
2003-07-23 & Clear & 4.08 & +0.04$\pm$0.02 &  38 &  16 &  -44$\pm$ 54 &  -12$\pm$ 45 &  -37$\pm$ 50 &   -6$\pm$ 45 &   -7$\pm$ 21 &   -6$\pm$  1 \\ 
2003-07-25 & Clear & 6.39 & -0.04$\pm$0.03 &  44 &  16 &  -60$\pm$138 &    9$\pm$ 50 &  -64$\pm$136 &    2$\pm$ 51 &    5$\pm$ 27 &    7$\pm$  2 \\ 
2003-07-26 & Clear & 6.97 & -0.04$\pm$0.04 &  33 &  14 & -133$\pm$167 &   21$\pm$ 90 & -121$\pm$165 &   13$\pm$ 91 &  -12$\pm$ 30 &    8$\pm$  5 \\ 
2003-07-27 & Clear & 1.54 & +0.01$\pm$0.05 &  26 &  14 &  -31$\pm$ 76 &   31$\pm$ 74 &  -40$\pm$ 76 &   34$\pm$ 74 &    9$\pm$  3 &   -3$\pm$  1 \\ 
2003-07-28 & Clear & 7.98 & -0.01$\pm$0.02 &  60 &  14 &  -51$\pm$130 &   46$\pm$133 &  -44$\pm$130 &   43$\pm$133 &   -6$\pm$ 10 &    3$\pm$  2 \\ 
2003-08-20 & CLEAR & 4.20 & +0.02$\pm$0.02 &  49 &  21 &   53$\pm$ 69 &   -8$\pm$ 45 &   46$\pm$ 68 &   -5$\pm$ 45 &    8$\pm$  9 &   -3$\pm$  1 \\ 
2003-10-14 & V     & 3.63 & -0.04$\pm$0.03 &  20 &  33 &   -4$\pm$ 54 &  -34$\pm$ 72 &   13$\pm$ 53 &  -40$\pm$ 74 &  -17$\pm$ 14 &    5$\pm$  2 \\ 
2003-10-15 & V     & 2.27 & +0.01$\pm$0.05 &  18 &  33 &  -22$\pm$ 38 &   -3$\pm$ 78 &  -23$\pm$ 38 &   -2$\pm$ 77 &    1$\pm$  1 &   -1$\pm$  0 \\ 
2003-10-16 & V     & 1.94 & -0.10$\pm$0.10 &   8 &  25 &  -18$\pm$ 64 &  -15$\pm$ 36 &   46$\pm$ 58 &  -32$\pm$ 32 &  -64$\pm$ 25 &   17$\pm$  6 \\ 
2003-10-17 & V     & 2.09 & +0.02$\pm$0.04 &  10 &  29 &    1$\pm$ 31 &    4$\pm$ 92 &   -7$\pm$ 31 &    7$\pm$ 92 &    8$\pm$  4 &   -2$\pm$  1 \\ 
2003-10-19 & V     & 1.89 & -0.12$\pm$0.05 &  12 &  31 &   -7$\pm$ 40 &  -45$\pm$ 35 &   44$\pm$ 32 &  -60$\pm$ 32 &  -51$\pm$ 23 &   15$\pm$  4 \\ 
2004-08-05 & V     & 2.38 & -0.06$\pm$0.02 &  21 &  15 &  -13$\pm$ 31 & -101$\pm$ 35 &   21$\pm$ 26 & -111$\pm$ 35 &  -34$\pm$ 17 &   10$\pm$  4 \\ 
2004-08-06 & V     & 3.27 & +0.06$\pm$0.07 &  22 &  16 &   53$\pm$ 99 & -110$\pm$ 50 &   45$\pm$ 97 & -102$\pm$ 50 &    8$\pm$ 21 &   -8$\pm$  2 \\ 
2004-08-07 & V     & 4.50 & +0.00$\pm$0.03 &  23 &  19 &   54$\pm$ 56 & -117$\pm$ 61 &   54$\pm$ 56 & -117$\pm$ 61 &    0$\pm$  1 &   -0$\pm$  0 \\ 
2004-08-20 & Clear & 3.81 & +0.03$\pm$0.01 &  16 &  24 &   -5$\pm$ 26 &  -40$\pm$ 24 &  -15$\pm$ 23 &  -35$\pm$ 23 &   11$\pm$ 13 &   -5$\pm$  2 \\ 
2004-08-21 & Clear & 4.09 & +0.02$\pm$0.02 &  27 &  22 &  -13$\pm$ 41 &  -32$\pm$ 52 &  -22$\pm$ 39 &  -28$\pm$ 53 &    9$\pm$ 11 &   -4$\pm$  1 \\ 
2004-08-21 & Clear & 2.57 & +0.01$\pm$0.02 &  26 &  21 &  -13$\pm$ 28 &  -51$\pm$ 35 &  -12$\pm$ 27 &  -51$\pm$ 35 &   -1$\pm$  2 &   -1$\pm$  0 \\ 
2004-08-23 & Clear & 5.20 & +0.02$\pm$0.01 &  43 &  18 &   -2$\pm$ 44 &  -36$\pm$ 49 &  -11$\pm$ 42 &  -32$\pm$ 49 &   10$\pm$ 11 &   -4$\pm$  3 \\ 
2004-08-24 & Clear & 3.94 & +0.02$\pm$0.01 &  29 &  17 &   -2$\pm$ 29 &  -34$\pm$ 62 &   -9$\pm$ 28 &  -31$\pm$ 62 &    8$\pm$  9 &   -3$\pm$  1 \\ 
2004-09-24 & R     & 3.08 & +0.04$\pm$0.04 &  37 &  14 &   26$\pm$111 &   83$\pm$131 &    2$\pm$109 &   91$\pm$130 &   24$\pm$ 19 &   -8$\pm$  4 \\ 
2004-09-25 & Clear & 3.75 & +0.01$\pm$0.04 &  36 &  14 &   59$\pm$113 &   91$\pm$ 77 &   51$\pm$113 &   93$\pm$ 77 &    7$\pm$  5 &   -2$\pm$  1 \\ 
2005-09-24 & V     & 2.78 & +0.01$\pm$0.01 & 155 &  16 &  -14$\pm$ 43 &  -95$\pm$ 56 &  -26$\pm$ 43 &  -91$\pm$ 56 &   11$\pm$  4 &   -3$\pm$  1 \\ 
2006-06-08 & Clear & 3.30 & +0.10$\pm$0.03 & 157 &  26 &  -22$\pm$104 &  -47$\pm$ 69 &    2$\pm$100 &  -32$\pm$ 69 &  -24$\pm$ 27 &  -15$\pm$  2 \\ 
2009-07-22 & Clear & 2.19 & +0.08$\pm$0.07 &  17 &  15 &   10$\pm$ 43 &   74$\pm$ 96 &   -1$\pm$ 41 &   87$\pm$ 95 &   10$\pm$ 12 &  -13$\pm$  0 \\ 
2011-09-05 & I     & 1.82 & +0.19$\pm$0.03 & 100 &  18 &   91$\pm$ 47 &   -1$\pm$ 34 &   36$\pm$ 38 &   38$\pm$ 36 &   55$\pm$ 28 &  -39$\pm$  3 \\ 
2011-09-26 & I     & 4.02 & -0.00$\pm$0.00 & 250 &  18 &   43$\pm$ 28 &    4$\pm$ 41 &   43$\pm$ 28 &    3$\pm$ 41 &   -1$\pm$  0 &    0$\pm$  0 \\ 
2012-10-19 & R     & 1.99 & +0.13$\pm$0.03 & 118 &   8 &   14$\pm$108 &  -76$\pm$134 &  -78$\pm$100 &  -38$\pm$136 &   93$\pm$ 39 &  -38$\pm$  9 \\ 
\hline
\end{longtable}

%\end{tabular}
%\label{Tab:netuno-160}
%\caption{Obtained parameters and offsets from adjustments of 160 images for Neptune.}
%\end{table*}

\end{landscape}


%Teoretically, the difference in the positions of the objects in the sense Triton - Neptune compared to the difference in the ephemeris over a night would cause the following effects.
%
%\begin{itemize}
%\item RA: the difference in the offsets would be positives in the East side of the sky, negative in the West side and zero in the culmination, with the assumption that only chromatic refraction affects the offsets.
%\item DEC: the difference in the offsets in the culmination would be positive (Neptune is in the North side in both nights for the site). Farther from the culmination the difference in the offsets would be more positive.
%\end{itemize}
%
%Fig \ref{Fig:refraction} clearly shows that that the chromatic refraction is affecting the offsets confirming our expectation. It is possible to see that the distribution of the offsets for the 5 night are very similar over the hour angles observed.
%
%The offsets corrected by chromatic refraction of the 5 nights is presented in Fig. \ref{Fig:refraction-cor}.
%
%\begin{figure}[h]
%\includegraphics[width=16.0cm]{plot_hour_dif_cor.png} 
%\caption{Same as in Fig. \ref{Fig:refraction} but corrected by chromatic refraction.}
%\label{Fig:refraction-cor}
%\end{figure}
%
%In Table \ref{Tab:refraction} it is shown the mean of the difference in the offsets of Triton and Neptune for each night and their dispersions. We made two tests in the chromatic refraction, the first one we made the correction in the difference of the offsets of both objects. In the second test we made the corrections in the offsets separately for each object, then we took the difference between them.
%
%\begin{table}[h]
%\centering
%\caption{Mean and Standard deviation of the difference in the offsets of Triton - Neptune before and after the chromatic refraction correction}
%\label{Tab:refraction}
%\begin{tabular}{|l|c|c|c|c|c|c|}
%\hline
% \multicolumn{1}{|c|}{Night} & \multicolumn{2}{|c|}{No correction} & \multicolumn{2}{|c|}{Correction 1} & \multicolumn{2}{|c|}{Correction 2}\\
%  & RA & DEC & RA & DEC & RA & DEC \\
% \hline
%PE:1997-05-31 &   83+-138 & -1+-34 &  58+-88 & -22+-34 & 58+- 88 & -22+-34 \\
%PE:1997-06-01 &   120+-118 &  31+-26 &  84+-38 &  9+-27 & 84+-38 &  9+-27 \\
%BC:2003-07-27  &  -13+-111 & 25+-70 &  32+-70 &  7+-70 & 35+-100 &  6+-70 \\
%BC:2004-08-22  &   -11+-43 & 53+-38 &  22+-22 & 41+-37 & 19+- 23 & 42+-37 \\
%BC:2011-09-25  &   -33+-34 & 70+-17 &   5+-23 & 55+-16  &  5+- 23 & 55+-16 \\
%\hline
%\end{tabular}
%"Correction 1" means that the correction was made in the difference of the offsets of Triton and Neptune. "Correction 2" means that the correction of chromatic refraction was made in the offsets of Triton and Neptune separately then we took the difference between them.
%\end{table}
%
%It is possible to see that the dispersion of the offsets in RA after the correction is much smaller than before the correction. The mean offsets in RA also show significant difference. For DEC the dispersion does not change, but the mean offsets presents significant difference. It is also possible to see that the results for both tests are basically the same, with the exception of the night of 2003 in RA.
%
%In Table \ref{Tab:refraction-cor} it is shown the values of $\Delta B$ obtained in the fit of the offsets of Triton and Neptune separately and their differences. The minimum and maximum values of Hour Angle for each night is also presented.
%
%\begin{table}[h]
%\centering
%\caption{Results of the fit of $\Delta B$ in the 5 nights for the difference Triton - Neptune and for each object separately.}
%\label{Tab:refraction-cor}
%\begin{tabular}{|c|c|c|c|c|c|c|c|}
%\hline
% Night & Filter  & $H_{min}$ & $H_{max}$ & $\Delta H$ & Object & $\Delta B$ & err $\Delta B$\\
% \hline
% & &  & & & T-N  &     0.214  &  0.004   \\
% PE:1997-05-31 &   Clear & -3.0 & 2.1 & 5.1 & Neptune &   -0.219 &  0.004  \\
% & & & & & Triton &  -0.006  &  0.004 \\
%\hline
% & & & & & T-N  &  0.239  &  0.004 \\
% PE:1997-06-01 & Clear & -2.6 & 2.2 & 4.8 & Neptune &   -0.347  &  0.004 \\
% & & & & & Triton &   -0.107  &   0.004  \\
%\hline
% & & & & & T-N &   0.052 &   0.003  \\
%BC:2003-07-27 & Clear & -3.5 & 4.8 & 8.3 & Neptune &   -0.015  &  0.002  \\
% & & & & & Triton &    0.038  &  0.003  \\
%\hline
% & & & & & T-N &   0.048  &  0.004  \\
%BC:2004-08-22 & Clear? & -0.8 & 4.3 & 5.1 & Neptune &   -0.059  &   0.003  \\
% & & & & & Triton &   -0.015  &  0.004 \\
%\hline
%& & & & & T-N &   0.046  &  0.002  \\
%BC:2011-09-25 & I & 0.4 & 4.5 & 4.1 & Neptune &   -0.046  &   0.002  \\
% & & & & & Triton &    0.000  &  0.002   \\
%\hline
%\end{tabular}
%\end{table}

%\section*{PSF for extended object Test}
%
%PRAIA uses a bidimensional circular gaussian fit for the PSF of the objects in the images to obtain their photocenters. But in the case of Neptune, its size is bigger than 2 arcsec, which is comparable to the observational seeing. In this case a gaussian fit may not represent with enough accuracy the PSF of Neptune.
%
%In order to obtain a PSF for extended objects (objects with sizes in the order of the seeing), Vieira-Martins (priv. comm.) obtained the following equation:
%
%\begin{equation}
%F_{c}(x,y) = A\sqrt{\frac{\pi}{2}}\sigma \int_{-R}^{+R} e^{-\frac{(x-x_{0}-\mu)^2}{2\sigma^2}} \left[ erf \left( \frac{y-y_{0}+\sqrt{R^2 - \mu^2}}{\sqrt{2}\sigma} \right) - erf \left( \frac{y-y_{0}-\sqrt{R^2 - \mu^2}}{\sqrt{2}\sigma} \right) \right] d\mu
%\end{equation}
%
%where erf(x) is the Error Function defined as:
%
%\begin{equation}
%erf(x) = \frac{2}{\sqrt{\pi}} \int_{0}^{x} e^{-t^2} dt
%\end{equation}
%
%The flux $F_c$ obtained for a given position $(x,y)$ is obtained by integrating an homogeneous disk where every bit of the disk can be represented by a bidimensional circular Gaussian fit. The parameters to be determined are the amplitude A and the dispersion $\sigma$ of the original Gaussian, the center of the disk ($x_0$, $y_0$) and the radius of the disk R.
%
%Fig. \ref{Fig:ajuste-psf} shows the first test with this PSF for a image of Neptune. The image test was taken at OPD in June 08, 2013. The image has a pixel scale of 0.177"/px, an exposure time of 5s and was observed with a B filter.
%
%\begin{figure}
%\includegraphics[width=16cm]{Ajuste.png} 
%\caption{Fit of an image of Neptune with a gaussian fit (red) and with the PSF for extended objects (green).\label{Fig:ajuste-psf}}
%\end{figure}
%
%In this example, the photocenter calculated was different from PRAIA's by 0.01px or 1.77mas. We must still estimate the best region of the image to fit the PSF. We must also test it for other nights.
%
%This test was made using the Scipy packages: Odrpack (for non-linear least square fit) and Special (for the error function). It is possible to see that the function for extended objects (green) fits better the profile of Neptune.

\section*{Results}

Finally, from the offsets in the sense "position minus ephemeris" it was made statistics night by night to eliminate discrepant positions with a sigma-clip procedure where offsets (modulus) larger than 80 mas or 2.5-sigma discrepant from the mean offset were removed.% This procedure was applied for each set of observations (short exposures and long exposures) separately.

Fig \ref{Fig:netuno-media} shows the mean ephemeris offsets of each night and respective discrepancy (error bars) for Neptune in RA and DEC, respectively.

%Fig \ref{Fig:triton-netuno-mean} shows the difference in the mean offsets night by night for all matched nights and not eliminated by the sigma-clip procedure in the sense Triton - Neptune. The error bars are the mean value of the dispersions in the night for each satellite.

%The large offsets found in 2002-2003 must be checked. It may be caused by few reference stars in the field or saturation.\\


%It seems that there are long term systematic errors in the orbit of Neptune, and in the orbit of Triton around Neptune, but it is too soon to state that with confidence. We must still further refine the positions.
%We still plan to do the following:
 %Figs. \ref{Fig:triton-netuno-1992}- \ref{Fig:triton-netuno-2013} shows the same divided by an interval of 3 years.

%\begin{itemize}
%\item Test the PSF for Neptune to identify the best way to reduce all images.
%\item Utilize an astrometry which uses the same stars in all fields of a night to have a cleaner reduction to eliminate chromatic refraction.
%\item Further refinements in the data may be needed as we further investigate these position sets.
%\end{itemize}

%In Tables 2 and 3 it is shown the mean offsets in Right Ascension and Declination night by night for Neptune and Triton, respectively, observed in the \PE telescope. The dispersion of the positions (standard deviation), number of frames that was not eliminated by the sigma-clip procedure, the mean date of the night and the average number of reference stars by frame is also available in the tables. The respective mean offsets night by night for the \BC telescope is available in Tables 4 and 5. As for the previous telescopes, tables 6 and 7 summarizes the offsets obtained with the Zeiss telescope.

 %The average dispersion of the offsets in Tables 2-7 is also presented in the table.


%\begin{figure}[h]
%\includegraphics[width=16.0cm]{Netuno_all.png} 
%\caption{Neptune - All Offsets}
%\label{Fig:netuno-all}
%\end{figure}
\begin{figure}[H]
\includegraphics[width=16.0cm]{Neptune_offsets.png} 
\caption{Neptune - Mean offsets by night. The plot shows the variation in the position of Neptune over time. }
\label{Fig:netuno-media}
\end{figure}
%\begin{figure}
%\includegraphics[width=16.0cm]{Triton_all.png} 
%\caption{Triton - All Offsets}
%\label{Fig:triton-all}
%\end{figure}
%\begin{figure}
%\includegraphics[width=16.0cm]{Triton_media.png} 
%\caption{Triton - Mean offsets by day}
%\label{Fig:trito-media}
%\end{figure}
%\begin{figure}
%\includegraphics[width=16.0cm]{Triton-Netuno_all.png} 
%\caption{Difference between the offsets of Triton and Neptune - All %data}
%\label{Fig:triton-netuno-all}
%\end{figure}
%\begin{figure}
%\includegraphics[width=15.0cm]{Triton-Neptune.png} 
%\caption{Difference between the offsets of Triton and Neptune - Mean offset by day}
%\label{Fig:triton-netuno-mean}
%\end{figure}

Fig. \ref{Fig:triton-netuno-anom} shows the difference between the ephemeris offsets of Triton and Neptune over the Argument of Latitude, so that in fact we get the "observed minus ephemeris" offsets of the relative positions of Triton with respect to Neptune. As shown by \cite{Emelyanov2015}, the movement of Triton around Neptune has a peculiarity where the True Anomaly, or Mean Anomaly, since Triton has an almost circular orbit, oscillates around zero with an amplitude of about 17\degr. \cite{Emelyanov2015} also shows that the Argument of Pericenter ($\omega$) of Triton's orbit rotates with an angular velocity approximately equal to the rotation of the satellite around the planet. These characteristics make the Argument of Latitude, sum of Mean Anomaly and Argument of Pericenter, more suitable to study the movement of Triton around Neptune.

\begin{figure}[H]
\includegraphics[width=16.0cm]{Anom_TN.png} 
\caption{Difference between the offsets of Triton and Neptune by Argument of Latitude. It shows the variation in the position of Triton around Neptune.}
\label{Fig:triton-netuno-anom}
\end{figure}

%\begin{figure}
%\includegraphics[width=14.0cm]{Anom_TN_1992-1995.png} 
%\caption{Difference between the offsets of Triton and Neptune by True Anomaly - 1992.0 -> 1995.0}
%\label{Fig:triton-netuno-1992}
%\end{figure}
%\begin{figure}
%\includegraphics[width=14.0cm]{Anom_TN_1995-1998.png} 
%\caption{Difference between the offsets of Triton and Neptune by True Anomaly - 1995.0 -> 1998.0}
%\label{Fig:triton-netuno-1995}
%\end{figure}
%\begin{figure}
%\includegraphics[width=14.0cm]{Anom_TN_1998-2001.png} 
%\caption{Difference between the offsets of Triton and Neptune by True Anomaly - 1998.0 -> 2001.0}
%\label{Fig:triton-netuno-1998}
%\end{figure}
%\begin{figure}
%\includegraphics[width=14.0cm]{Anom_TN_2001-2004.png} 
%\caption{Difference between the offsets of Triton and Neptune by True Anomaly - 2001.0 -> 2004.0}
%\label{Fig:triton-netuno-2001}
%\end{figure}
%\begin{figure}
%\includegraphics[width=14.0cm]{Anom_TN_2004-2007.png} 
%\caption{Difference between the offsets of Triton and Neptune by True Anomaly - 2004.0 -> 2007.0}
%\label{Fig:triton-netuno-2004}
%\end{figure}
%\begin{figure}
%\includegraphics[width=14.0cm]{Anom_TN_2007-2010.png} 
%\caption{Difference between the offsets of Triton and Neptune by True Anomaly - 2007.0 -> 2010.0}
%\label{Fig:triton-netuno-2007}
%\end{figure}
%\begin{figure}
%\includegraphics[width=14.0cm]{Anom_TN_2010-2013.png} 
%\caption{Difference between the offsets of Triton and Neptune by True Anomaly - 2010.0 -> 2013.0}
%\label{Fig:triton-netuno-2010}
%\end{figure}
%\begin{figure}
%\includegraphics[width=14.0cm]{Anom_TN_2013-2016.png} 
%\caption{Difference between the offsets of Triton and Neptune by True Anomaly - 2013.0 -> 2016.0}
%\label{Fig:triton-netuno-2013}
%\end{figure}

\section*{Final tests}

In order to consolidate the position data sets of Neptune and Triton prior to publication, we have two tests yet to perform.


a) Numerical PSF for Neptune:

Roberto Vieira Martins developed a numerical PSF for a circular extended object, based on the spread of a Gaussian PSF kernel over the apparent planet's disk. The effects of solar phase angle were not considered yet in this first numerical PSF model. I've already implemented this numerical PSF. I am testing it in a group of observations with negligible solar phase angle. The idea is to compare the results using this numerical PSF and the 2D Gaussian PSF, which was used in all observations.

If no significant improvements are achieved, we will stand with the 2D Gaussian PSF. This may be the case, as we are dealing with high S/N ratio images (Neptune) and PRAIA discards wing pixels in the (x,y) fits, so that in the end both PSFs may perform equally well. If the offsets of the observations tested are improved, the next step is to upgrade the PSF for observations with higher phase angle, and then apply the model in all Neptune measurements.


b) Uniform reduction and global reduction:

We will make tests on selected nights, using rigorously the same reference stars in the (RA,Dec) reduction of each individual field of that night (uniform reduction), or using the global reduction technique with the same underlying objective, in order to see if we get a better chromatic refraction correction.


If we do not find any position improvements in tests (a) and (b), then the current sets of positions will be the final ones. If improvements can be made from the tests, then we will apply the procedures in the reductions in order to have the final sets of positions. 


\bibliography{references}
\bibliographystyle{apalike}


\end{document}
